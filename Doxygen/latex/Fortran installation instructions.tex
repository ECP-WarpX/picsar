\section*{I\+N\+S\+T\+A\+L\+L\+I\+NG P\+I\+C\+S\+AR IN F\+U\+LL F\+O\+R\+T\+R\+AN 90 }

\subsection*{Makefile config }

First edit the file Makefile and indicate the following environment variables\+:

FC\+: your M\+PI Fortran compiler wrapper (e.\+g mpif90, mpiifort, ftn etc.),

F\+A\+R\+GS\+: your compiler arguments (optimization flags etc.). To get Open\+MP version of P\+I\+C\+S\+AR use the flag -\/fopenmp (with gfortran) and -\/openmp (Cray, Intel). NB\+: this version of P\+I\+C\+S\+AR requires at least {\bfseries Open\+MP 4.\+0}.

\subsection*{Compiling }

To compile, invoke the rule \char`\"{}all\char`\"{}\+:

Make -\/f Makefile all

Testing 



Testing the code after compilation is highly recommended.

To test the compilation/execution, you can use the makefile (py.\+test is required)\+:

For all tests\+: \begin{quote}
make -\/f Makefile test \end{quote}


For each test one by one
\begin{DoxyItemize}
\item Drifted plasma\+: make -\/f Makefile test1
\item Homogeneous plasma\+: make -\/f Makefile test2
\item Langmuir wave\+: make -\/f Makefile test3
\end{DoxyItemize}

To test the compilation/execution, you can run test python scripts manually\+:

Go the folder Acceptance\+\_\+testing/\+Fortran\+\_\+tests.

You can perform each test one by one entering the different folders\+:
\begin{DoxyItemize}
\item test\+\_\+plasma\+\_\+drift
\item test\+\_\+homogeneous\+\_\+plasma
\item test\+\_\+\+Langmuir\+\_\+wave
\end{DoxyItemize}

You can run scripts using py.\+test\+: \begin{quote}
py.\+test -\/s --trun=1 --ttest=1 \end{quote}


--trun=0/1\+: this option enables/disables simulation run --ttest=0/1\+: this option enables/disables assert tests

You can run scripts without py.\+test\+: \begin{quote}
python $<$pyhthon\+\_\+script$>$ -\/r 1 -\/t 1 \end{quote}


-\/r 0/1\+: this option enables/disables simulation run -\/t 0/1\+: this option enables/disables simulation assert tests 