\section*{Input file Configuration }

The input file is divided into sections. Sections start by {\ttfamily section\+::name} where {\ttfamily name} is the section name and end with {\ttfamily end\+::name}.

\paragraph*{A. cpusplit section}

This section enables to configure the M\+PI decomposition and other parameters\+:


\begin{DoxyItemize}
\item {\ttfamily nprocx}, {\ttfamily nprocy}, {\ttfamily nprocz}\+: number of processors in each direction x, y, z
\item {\ttfamily topology}\+: the M\+PI topology, 0 corresponds to cartesian
\end{DoxyItemize}

\paragraph*{B. main section}

This section enables to configure the general simulation parameters\+:


\begin{DoxyItemize}
\item {\ttfamily nx}, {\ttfamily ny}, {\ttfamily nz}\+: number of grid points (domain discretization) in each direction
\item {\ttfamily xmin}, {\ttfamily xmax}, {\ttfamily ymin}, {\ttfamily ymax}, {\ttfamily zmin}, {\ttfamily zmax}\+: Origin of simulation axes
\item {\ttfamily t\+\_\+max}\+: final time (0 by default)
\item {\ttfamily nsteps}\+: can be specified instead of the final time, else nsteps is determine from the final time and the C\+FL condition (0 by default)
\item {\ttfamily ntilex}, {\ttfamily ntiley}, {\ttfamily ntilez}\+: M\+PI sub-\/domain discetization into tiles
\item {\ttfamily nguardsx}, {\ttfamily nguardsy}, {\ttfamily nguardsz}\+: guard cells for the field arrays
\item {\ttfamily njguardsx}, {\ttfamily njguardsy}, {\ttfamily njguardsz}\+: guard cells for the current arrays
\end{DoxyItemize}

\paragraph*{C. solver section}

This section, {\ttfamily section\+::solver}, enables to controle the solver and algorithm parameters\+:


\begin{DoxyItemize}
\item {\ttfamily norderx}, {\ttfamily nordery}, {\ttfamily norderz}\+: Maxwell solver orders
\item {\ttfamily nox}, {\ttfamily noy}, {\ttfamily noz}\+: shape factor (interpolation) orders, note that optimized subroutines only work when {\ttfamily nox=noy=noz}
\item {\ttfamily currdepo}\+: current deposition algorithm
\begin{DoxyItemize}
\item {\ttfamily =0}\+: Esirkepov with tiling/\+Open\+MP and optimized for A\+V\+X512. For the moment, in 3D, only {\ttfamily nox=noy=noz=1} provides better performances.
\item {\ttfamily =1}\+: Esirkepov with tiling/\+Open\+MP and non-\/optimized. The functions provided for {\ttfamily nox=noy=noz} are much faster than using different orders (in this case, an arbitrary order subroutine with many if-\/statements is used).
\item {\ttfamily =2}\+: Esirkepov sequential
\item {\ttfamily =3}\+: Classical current deposition with Tiling/\+Open\+MP and optimized/vectorized subroutines. This provides the best performance even with A\+VX architectures.
\item {\ttfamily =4}\+: Classical current deposition with Tiling/open\+MP and non-\/optimized subroutines.
\item {\ttfamily =5}\+: Classical current deposition sequential
\end{DoxyItemize}
\item {\ttfamily fieldgathe}\+: field gathering
\begin{DoxyItemize}
\item {\ttfamily =0}\+: specific order vectorized subroutine when {\ttfamily nox=noy=noz}
\item {\ttfamily =1}\+: specific order scalar subroutines when {\ttfamily nox=noy=noz}
\item {\ttfamily =2}\+: arbitrary order non-\/optimized subroutines (W\+A\+RP original)
\end{DoxyItemize}
\item {\ttfamily fg\+\_\+p\+\_\+pp\+\_\+seperated}\+: field gathering + particle pusher
\begin{DoxyItemize}
\item {\ttfamily =0}\+: field gathering + particle pusher in the same loop
\item {\ttfamily =1}\+: field gathering + particle pusher in the same tile
\item {\ttfamily =2}\+: field gathering + particle pusher separated
\end{DoxyItemize}
\item {\ttfamily rhodepo}\+: charge deposition
\begin{DoxyItemize}
\item {\ttfamily =0}\+: specific order vectorized subroutines when {\ttfamily nox=noy=noz}
\item {\ttfamily =1}\+: specific order scalar subroutines when {\ttfamily nox=noy=noz}
\item {\ttfamily =2}\+: arbitrary order non-\/optimized subroutines (W\+A\+RP original)
\end{DoxyItemize}
\item {\ttfamily partcom}\+: particle communications
\begin{DoxyItemize}
\item {\ttfamily =0}\+: Communications between tiles and between M\+PI domains is done in the same subroutine (overlapped computation) in parallel
\item {\ttfamily =1}\+: Communications are done separately with Open\+MP for the preprocessing loop
\item {\ttfamily =2}\+: Communications are done separately without Open\+MP
\end{DoxyItemize}
\item {\ttfamily lvec\+\_\+curr\+\_\+depo}\+: vector block length for the current deposition (8 by default)
\item {\ttfamily lvec\+\_\+charge\+\_\+depo}\+: vector block length for the charge deposition (64 by default)
\item {\ttfamily lvec\+\_\+fieldgathe}\+: vector block length (field gathering cache blocking) for the field gathering (256 by default)
\item {\ttfamily mpi\+\_\+buf\+\_\+size}\+: size of the mpi buffers for the particle communications (2000 by default)
\end{DoxyItemize}

\paragraph*{D. Plasma section}

This section, {\ttfamily section\+::plasma}, enables to controle the plasma parameters\+:


\begin{DoxyItemize}
\item {\ttfamily nlab}\+: density in the laboratory
\item {\ttfamily pdistr}\+: initial distribution
\begin{DoxyItemize}
\item {\ttfamily =1}\+: ordered space initialization
\item {\ttfamily =2}\+: random space initialization
\end{DoxyItemize}
\end{DoxyItemize}

\paragraph*{E. Species section}

This section, {\ttfamily section\+::species}, enables to configure the species properties. It has to be repeated for each species.


\begin{DoxyItemize}
\item {\ttfamily name}\+: species name
\item {\ttfamily mass}\+: species mass normalized to the electronic mass
\item {\ttfamily charge}\+: species charge normalized to the positron charge
\item {\ttfamily nppcell}\+: number of particles per cell
\item {\ttfamily x\+\_\+min}, {\ttfamily x\+\_\+max}, {\ttfamily y\+\_\+min}, {\ttfamily y\+\_\+max}, {\ttfamily z\+\_\+min}, {\ttfamily z\+\_\+max}\+: plasma expansion in each direction
\item {\ttfamily vdrift\+\_\+x}, {\ttfamily vdrift\+\_\+y}, {\ttfamily vdrift\+\_\+z}\+: drift velocity in each direction
\item {\ttfamily vth\+\_\+x}, {\ttfamily vth\+\_\+y}, {\ttfamily vth\+\_\+z}\+: thermal velocity in each direction
\item {\ttfamily sorting\+\_\+period}\+: period of the sorting
\item {\ttfamily sorting\+\_\+start}\+: beginning of the sorting
\end{DoxyItemize}

\paragraph*{F. Sorting section}

This section, {\ttfamily section\+::sorting}, enables to controle the particle cell sorting algorithm.


\begin{DoxyItemize}
\item {\ttfamily activation}\+: activation of the sorting
\item {\ttfamily dx}, {\ttfamily dy}, {\ttfamily dz}\+: size of the sorting cells
\item {\ttfamily shiftx}, {\ttfamily shifty}, {\ttfamily shiftz}\+: shift of the sorting grid
\end{DoxyItemize}

\paragraph*{G. Time statistic section}

This section, {\ttfamily section\+::timestat}, enables to controle the time statistics information. All main subroutines have internal timers. The results of time spent in each part of the code can be displayed in the terminal or write in a file. By default, the time statistics are given at the end of the simulation in the terminal.


\begin{DoxyItemize}
\item {\ttfamily activation} ({\ttfamily =0/1})\+: this flag activates output of the time statistics as a function of the iteration number in a file
\item {\ttfamily period}\+: this is the period of the output in number of iterations
\end{DoxyItemize}

\paragraph*{H. Temporal diagnostics section}