\section*{I\+N\+S\+T\+A\+L\+L\+A\+T\+I\+ON OF P\+Y\+T\+H\+ON M\+O\+D\+U\+LE P\+I\+C\+S\+AR }

\subsection*{1. Installing python and packages }

Install python 2 or later. We recommend python from anaconda ({\ttfamily \href{http://docs.continuum.io/anaconda/install}{\tt http\+://docs.\+continuum.\+io/anaconda/install}}) Install numpy (pip install numpy) Install mpi4py (pip install mpi4py)

\subsection*{2. Installing Forthon }

Before creating the python module picsarpy for picsar, you must install the Forthon compiler. To do so\+:

Copy the last stable version of Forthon by typing\+: {\ttfamily git clone \href{https://github.com/dpgrote/Forthon.git}{\tt https\+://github.\+com/dpgrote/\+Forthon.\+git}}

Follow installation steps detailed in R\+E\+A\+D\+ME

\subsection*{3. Makefile\+\_\+\+Forthon config }

First edit the file Makefile\+\_\+\+Forthon and indicate the following environment variables\+:


\begin{DoxyItemize}
\item F\+C\+O\+MP\+: your fortran compiler (e.\+g gfortran),
\item F\+C\+O\+M\+P\+E\+X\+EC\+: your M\+PI Fortran wrapper (e.\+g mpif90),
\item F\+A\+R\+GS\+: arguments of the  compiler. To get Open\+MP version of P\+I\+C\+S\+AR use the flag -\/fopenmp (with gfortran) and -\/openmp (Cray, Intel). NB\+: this version of P\+I\+C\+S\+AR requires at least {\bfseries Open\+MP 4.\+0}.
\item L\+I\+B\+D\+IR\+: your library folder containing M\+PI libraries (e.\+g /usr/local/\+Cellar/open-\/mpi/1.8.\+6/lib/ for an Homebrew install of open-\/mpi on M\+A\+C\+O\+SX, /opt/local/lib/mpich-\/mp/ for a Macports install of mpich),
\item L\+I\+BS\+: required libraries for the install. With Open-\/\+M\+PI, the compilation of picsar requires the following libraries\+: -\/lmpi, -\/lmpi\+\_\+usempi, -\/lmpi\+\_\+mpifh, -\/lgomp. For open-\/mpi$>$1.\+8.\+x, you should use -\/lmpi\+\_\+usempif08 instead of -\/lmpi\+\_\+usempi. For a Macports install of mpich, you should use -\/lmpifort -\/lmpi -\/lpmpi.
\end{DoxyItemize}

\subsection*{4. Compiling }

To compile and test, invoke the rule \char`\"{}all\char`\"{}\+:


\begin{DoxyItemize}
\item Make -\/f Makefile\+\_\+\+Forthon all
\end{DoxyItemize}

\subsection*{5. Compiling }

Testing the code after compilation is highly recommended.

To test the compilation/execution, you can use the makefile (py.\+test is required)\+:

For all test\+:
\begin{DoxyItemize}
\item Make -\/f Makefile\+\_\+\+Forthon test
\end{DoxyItemize}

For each test one by one
\begin{DoxyItemize}
\item Simple running test\+: make -\/f Makefile\+\_\+\+Forthon test1
\item Langmuir wave\+: make -\/f Makefile\+\_\+\+Forthon test2 
\end{DoxyItemize}