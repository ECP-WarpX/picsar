
\usepackage{bm}
\usepackage{amsmath}
\usepackage{amssymb}
\usepackage{graphicx}
\usepackage{url}
\usepackage{hyperref}

\usepackage[displaymath]{lineno}\usepackage{bm}% bold math

\newcommand{\fe}{\mathbf{\tilde{E}}}
\newcommand{\fb}{\mathbf{\tilde{B}}}
\newcommand{\fj}{\mathbf{\tilde{J}}}
\newcommand{\ff}{\tilde{F}}
\newcommand{\fg}{\tilde{G}}
\newcommand{\fk}{\mathbf{k}}
\newcommand{\fkhat}{\mathbf{\hat{k}}}

% Definitions from Remi's paper on Galilean math
\newcommand{\Km}{\vec{K}_{\vec{m}}}
\newcommand{\km}{\vec{k}_{\vec{m}}}
\renewcommand{\vec}[1]{\boldsymbol{#1}}
\newcommand{\vgal}{\vec{v}_{gal}}
\newcommand{\nab}{\vec{\nabla'}}
\newcommand{\Dt}[1]{ \frac{\partial #1}{\partial t}}
\newcommand{\mc}[1]{\hat{\mathcal{#1}}}
\newcommand{\xj}{\vec{x}'_{\vec{j}}}
\newcommand{\Xll}{\vec{X}_{\vec{\ell}}}
\newcommand{\Integ}[1]{\int_{-\infty}^{\infty} \!\!\!\!\!\!
  \mathrm{d}#1}
\newcommand{\RInteg}[1]{\int_{0}^{\infty} \!\! \frac{#1\mathrm{d}#1}{(2\pi)^2}}

% Definitions from Remi's Thesis
\newcommand{\h}{\mathcal{H}}
\newcommand{\hf}{\frac{1}{2}}
\newcommand{\um}{$\mu$m}
\newcommand{\Um}{\mu \mathrm{m}}
\newcommand{\aal}{\langle \vec{a}_l^2 \rangle}
\newcommand{\etad}{ \eta_d }
\newcommand{\etae}{ \eta_\epsilon }
\newcommand{\etag}{ \eta_\gamma }
\newcommand{\tlambda}{ \tilde{\lambda} }
%\newcommand\comment[1]{\textcolor{red}{\textbf{#1}}}
\newcommand{\gsim}{\mathrel{\hbox{\rlap{\lower.55ex 
\hbox{$\sim$}} \kern-.3em \raise.4ex \hbox{$>$}}}}
\newcommand{\lsim}{\mathrel{\hbox{\rlap{\lower.55ex 
\hbox{$\sim$}} \kern-.3em \raise.4ex \hbox{$<$}}}}
\newcommand{\kfoc}{k_\mathrm{foc}}
\newcommand{\bkfoc}{\bar{k}_\mathrm{foc}}
\newcommand{\xil}{\xi_{\mathrm{laser}}}

\newcommand{\Ex}[2]{{E_x}^{#1}_{#2}}
\newcommand{\Ey}[2]{{E_y}^{#1}_{#2}}
\newcommand{\Ez}[2]{{E_z}^{#1}_{#2}}
\newcommand{\Bx}[2]{{B_x}^{#1}_{#2}}
\newcommand{\By}[2]{{B_y}^{#1}_{#2}}
\newcommand{\Bz}[2]{{B_z}^{#1}_{#2}}
\newcommand{\Jx}[2]{{J_x}^{#1}_{#2}}
\newcommand{\Jy}[2]{{J_y}^{#1}_{#2}}
\newcommand{\Jz}[2]{{J_z}^{#1}_{#2}}

\newcommand{\tEr}[2]{\tilde{E_r}^{#1}_{#2}}
\newcommand{\tEt}[2]{\tilde{E_\theta}^{#1}_{#2}}
\newcommand{\tEz}[2]{\tilde{E_z}^{#1}_{#2}}
\newcommand{\tBr}[2]{\tilde{B_r}^{#1}_{#2}}
\newcommand{\tBt}[2]{\tilde{B_\theta}^{#1}_{#2}}
\newcommand{\tBz}[2]{\tilde{B_z}^{#1}_{#2}}
\newcommand{\tJr}[2]{\tilde{J_r}^{#1}_{#2}}
\newcommand{\tJt}[2]{\tilde{J_\theta}^{#1}_{#2}}
\newcommand{\tJz}[2]{\tilde{J_z}^{#1}_{#2}}

\newcommand{\CCirc}{\textsc{Calder Circ}}
\newcommand{\CCart}{\textsc{Calder 3D}}

Although full PIC codes are powerful tools, which capture a wide range
of physical phenomena, they also require large computational ressources.
This is partly due to the use of a 3D Cartesian grid, which
leads to a very large number of grid cells. (Typical 3D simulations of
laser-wakefield acceleration require $\sim 10^6$--$ 10^8$ grid
cells.) For this reason, these algorithms need to be highly parallelized, and
high-resolution simulations can only be run on costly large-scale
computer facilities. However, when the driver is
cylindrically-symmetric, it is possible to take advantage of the
symmetry of the problem to reduce the computational cost of the algorithm \cite{godfrey1985iprop,LifschitzJCP2009,DavidsonJCP2015,Lehe2016}.

\subsection{Azimuthal decomposition}
Let us consider the fields $\vec{E}$, $\vec{B}$, $\vec{J}$ and $\rho$
 in cylindral coordinates $(r,\theta,z)$, expressed as a Fourier series in $\theta$:
%
\begin{equation}
F(r,\theta,z) = \mathrm{Re}\left[ \sum_{\ell=0}^\infty
  \tilde{F}_{\ell}(r,z) e^{-i\ell\theta} \right]
\label{eq:chap2:azimuthal}
\end{equation}
%
\begin{equation}
\mathrm{with} \qquad \tilde{F}_{\ell} = C_\ell \int_0^{2\pi} d\theta
\,F(r,\theta,z)e^{i\ell\theta} \qquad
\label{eq:chap2:Fourier-coeffs}
\end{equation}
\begin{equation}
\mathrm{and} \;
\left \{ \begin{array}{l l}
C_{0} = 1/2\pi &\\
C_\ell = 1/\pi &\mathrm{for}\,\ell > 0
\end{array} \right.
\end{equation}
%
where $F$ represents any of the quantities $E_r$,
$E_\theta$, $E_z$, $B_r$, $B_\theta$, $B_z$, $J_r$, $J_\theta$, $J_z$
are $\rho$, and where the
$\tilde{F}_\ell$ are the associated Fourier components ($\ell$ is the
index of the corresponding azimuthal mode). In the general case, this
azimuthal decomposition does not simplify the problem, since an
infinity of modes have to be considered in (\ref{eq:chap2:azimuthal}). However, in the case of a
cylindrically-symmetric laser pulse, only the very first modes have
non-zero components. For instance, the wakefield is represented
exclusively by the mode $\ell = 0$. (This is because the quantities $E_r$,
$E_\theta$, $E_z$, $B_r$, $B_\theta$, $B_z$, $J_r$, $J_\theta$, $J_z$
and $\rho$ associated with the
wakefield are independent of $\theta$.) On the other hand, the field
of the laser pulse \emph{does}
depend on $\theta$, in cylindrical coordinates. For example, for a
cylindrically-symmetric pulse propagating along $z$ and polarized along $\vec{e}_\alpha = \cos(\alpha)\vec{e}_x + \sin(\alpha)\vec{e}_y$:
%
\begin{align}
\vec{E} &= E_0(r,z)\vec{e}_\alpha \\
& = E_0(r,z) [\; \cos(\alpha)(\cos(\theta)\vec{e}_r - \sin(\theta)\vec{e}_\theta) \; \nonumber \\
& + \; \sin(\alpha)(\sin(\theta)\vec{e}_r + \cos(\theta)\vec{e}_\theta) \; ]\\
& = \mathrm{Re}[ \; E_0(r,z) e^{i\alpha} e^{-i\theta} \; ]\vec{e}_r \; \nonumber \\
& + \; \mathrm{Re}[ \; -i E_0(r,z) e^{i\alpha} e^{-i\theta} \; ]\vec{e}_\theta.
\end{align}
%
Here the amplitude $E_0$ does not depend on $\theta$ because the pulse was assumed
to be cylindrically symmetric. In this case, the above relation shows
that the fields $E_r$ and $E_\theta$ of the laser are represented
exclusively by the mode $\ell = 1$. A similar calculation shows that
the same holds for $B_r$ and $B_\theta$. On the whole, only the modes
$\ell = 0$ and $\ell = 1$ are a priori necessary to model
laser-wakefield acceleration.
Under those conditions, the infinite sum in
(\ref{eq:chap2:azimuthal}) is truncated at a chosen $\ell_{max}$. In
principle, $\ell_{max} = 1$ is sufficient for laser-wakefield
acceleration. However, $\ell_{max}$ is kept as a free parameter in the algorithm, in order to verify that
higher modes are negligible, as well as to allow for less-symmetric configurations.
Because codes based on this algorithm are able to take into account the modes with $\ell > 0$, they are said to be
``quasi-cylindrical'' (or ``quasi-3D'' by some authors \cite{DavidsonJCP2015}), in contrast to cylindrical codes, which
assume that all fields are independent of $\theta$, and thus only
consider the mode $\ell = 0$.

\subsection{Discretized Maxwell equations} When the Fourier expressions
of the fields are injected into the Maxwell equations (written in
cylindrical coordinates), the different azimuthal modes
decouple. In this case, the Maxwell-Amp\`ere and Maxwell-Faraday equations
-- which are needed to update the fields in the PIC cycle -- can be written separately
for each azimuthal mode $\ell$:
\begin{subequations}
\begin{align}
\frac{\partial \tilde{B}_{r,\ell} }{\partial t} &=
\frac{i\ell}{r}\tilde{E}_{z,\ell} + \frac{\partial
  \tilde{E}_{\theta,\ell}}{\partial z} \\[3mm]
\frac{\partial \tilde{B}_{\theta,\ell} }{\partial t} &=
 - \frac{\partial \tilde{E}_{r,\ell}}{\partial z} + \frac{\partial
  \tilde{E}_{z,\ell}}{\partial r} \\[3mm]
\frac{\partial \tilde{B}_{z,\ell} }{\partial t} &=
- \frac{1}{r} \frac{\partial (r\tilde{E}_{\theta,\ell})}{\partial r} - \frac{i\ell}{r}\tilde{E}_{r,\ell} \\[3mm]
\frac{1}{c^2} \frac{\partial \tilde{E}_{r,\ell} }{\partial t} &=
-\frac{i\ell}{r}\tilde{B}_{z,\ell} - \frac{\partial
  \tilde{B}_{\theta,\ell}}{\partial z} - \mu_0 \tilde{J}_{r,\ell} \\[3mm]
\frac{1}{c^2}\frac{\partial \tilde{E}_{\theta,\ell} }{\partial t} &=
 \frac{\partial \tilde{B}_{r,\ell}}{\partial z} - \frac{\partial
  \tilde{B}_{z,\ell}}{\partial r} - \mu_0 \tilde{J}_{\theta,\ell} \\[3mm]
\frac{1}{c^2}\frac{\partial \tilde{E}_{z,\ell} }{\partial t} &=
 \frac{1}{r} \frac{\partial (r\tilde{B}_{\theta,\ell})}{\partial r} +
 \frac{i\ell}{r}\tilde{B}_{r,\ell} - \mu_0 \tilde{J}_{z,\ell}
\end{align}
\end{subequations}
%\begin{figure}
%\input{./Chap2/Circ_lattice.tex}
%\caption{Representation of the lattice in \CCirc. The
%  table shows at which
%  position each component of the fields is defined ($j$,$k$ and
%  $n$ are integers ; $\Delta r$ and $\Delta z$ are the
%  spatial steps of the grid). The above sketch represents one grid
%  cell, and the positions of the fields within it.}
%\label{fig:chap2:Circ_lattice}
%\end{figure}
In order to discretize these equations, each azimuthal mode is
represented on a two-dimensional grid,
%(The two dimensions correspond
%to $r$ and $z$.) \Cref{fig:chap2:Circ_lattice} summarizes the
%positions of the different fields within one grid cell, as well as the
%corresponding notations for these fields. Using these notations,
on which the discretized
Maxwell-Amp\`ere and Maxwell-Faraday equations are given by
%\begin{strip}
%\begin{align*}
%
%\frac{ \tBr{n+\hf}{j,\ell,k+\hf}- \tBr{n-\hf}{j,\ell,k+\hf}
%}{\Delta t} =& \frac{i\,\ell}{j\Delta r}\tEz{n}{j,\ell,k+\hf} + (D_z \tilde{E}_{\theta}^n)_{j,\ell,k+\hf} \\
%
%\frac{ \tBt{n+\hf}{j+\hf,\ell,k+\hf}- \tBt{n-\hf}{j+\hf,\ell,k+\hf} }{\Delta t} =& -(D_z \tilde{E}_r^n)_{j+\hf,\ell,k+\hf} + (D_r \tilde{E}_z^{n})_{j+\hf,\ell,k+\hf} \\
%
%\frac{ \tBz{n+\hf}{j+\hf,\ell,k}- \tBz{n-\hf}{j+\hf,\ell,k} }{\Delta t} =&
% -\frac{(j+1)\tEt{n}{j+1,\ell,k} - j\tEt{n}{j,\ell,k}}{(j+\hf)\Delta r} -
%\frac{i\,\ell}{(j+\hf) \Delta r}\tEr{n}{j+\hf,\ell,k} \\
%
%\frac{ \tEr{n+1}{j+\hf,\ell,k}- \tEr{n}{j+\hf,\ell,k}}{c^2 \Delta t}
%=& -\frac{i\,\ell}{(j+\hf)\Delta r}\tBz{n+\hf}{j+\hf,\ell,k} - (D_z \tilde{B}_{\theta}^{n+\hf})_{j+\hf,\ell,k} - \mu_0\tJr{n+\hf}{j+\hf,\ell,k} \\
%
%\frac{ \tEt{n+1}{j,\ell,k}- \tEt{n}{j,\ell,k}}{c^2 \Delta t}
%=& (D_z \tilde{B}_r^{n+\hf})_{j,\ell,k} - (D_r \tilde{B}_z^{n+\hf})_{j,\ell,k} - \mu_0\tJt{n+\hf}{j,\ell,k} \\
%
%\frac{ \tEz{n+1}{j,\ell,k+\hf}- \tEz{n}{j,\ell,k+\hf}}{c^2 \Delta t}
%=&   \frac{\left(j+\hf\right)\tBt{n+\hf}{j+\hf,\ell,k+\hf}  - \left(j-\hf\right)\tBt{n+\hf}{j-\hf,\ell,k+\hf}}{j\Delta r} \\
%& \qquad  \qquad + \frac{i\,\ell}{j\Delta r}\tBr{n+\hf}{j,\ell,k+\hf} - \mu_0\tJz{n+\hf}{j,\ell,k+\hf}
%\end{align*}
%\end{strip}

\begin{subequations}
\begin{align}
%
D_{t}\tilde{B}_r|_{j,\ell,k+\hf}^{n} \nonumber
=& \frac{i\,\ell}{j\Delta r}\tEz{n}{j,\ell,k+\hf} \\
& + D_z \tilde{E}_{\theta}|^n_{j,\ell,k+\hf} \\
%
D_{t}\tilde{B}_\theta|_{j+\hf,\ell,k+\hf}^{n} \nonumber
=& -D_z \tilde{E}_r|^n_{j+\hf,\ell,k+\hf} \\
& + D_r \tilde{E}_z|^{n}_{j+\hf,\ell,k+\hf} \\
%
D_{t}\tilde{B}_z|_{j+\hf,\ell,k}^{n} =& \nonumber
 -\frac{(j+1)\tEt{n}{j+1,\ell,k} }{(j+\hf)\Delta r} \\ \nonumber
 & +\frac{ j\tEt{n}{j,\ell,k}}{(j+\hf)\Delta r} \\
 & - \frac{i\,\ell}{(j+\hf) \Delta r}\tEr{n}{j+\hf,\ell,k}
\end{align}
\end{subequations}
for the magnetic field components, and
\begin{subequations}
\begin{align}
%
\frac{1}{c^2}D_{t}\tilde{E}_r|_{j+\hf,\ell,k}^{n+\hf} \nonumber
=& -\frac{i\,\ell}{(j+\hf)\Delta r}\tBz{n+\hf}{j+\hf,\ell,k} \\
& - D_z \tilde{B}_{\theta}|^{n+\hf}_{j+\hf,\ell,k} \nonumber\\
& - \mu_0\tJr{n+\hf}{j+\hf,\ell,k} \\
%
\frac{1}{c^2}D_{t}\tilde{E}_\theta|_{j,\ell,k}^{n+\hf} \nonumber
=& D_z \tilde{B}_r|^{n+\hf}_{j,\ell,k} - D_r \tilde{B}_z|^{n+\hf}_{j,\ell,k} \\
& - \mu_0\tJt{n+\hf}{j,\ell,k} \\
%
\frac{1}{c^2}D_{t}\tilde{E}_z|_{j,\ell,k+\hf}^{n+\hf} \nonumber
=&   \frac{\left(j+\hf\right)\tBt{n+\hf}{j+\hf,\ell,k+\hf} }{j\Delta r} \\
=&   -\frac{\left(j-\hf\right)\tBt{n+\hf}{j-\hf,\ell,k+\hf}}{j\Delta r} \nonumber\\
& + \frac{i\,\ell}{j\Delta r}\tBr{n+\hf}{j,\ell,k+\hf} \nonumber\\
& - \mu_0\tJz{n+\hf}{j,\ell,k+\hf}
\end{align}
\end{subequations}
for the electric field components.

The numerical operator $D_r$ and $D_z$ are defined by
\begin{align*}
(D_r F)_{j',\ell,k'} = \frac{F_{j'+\hf,\ell,k'}-F_{j'-\hf,\ell,k'} }{\Delta r} \\
(D_z F)_{j',\ell,k'} = \frac{F_{j',\ell,k'+\hf}-F_{j',\ell,k'-\hf} }{\Delta z} \\
\end{align*}
where $j'$ and $k'$ can be integers or half-integers. Notice
that these discretized Maxwell equations are not valid on-axis (i.e. for $j=0$), due to
singularities in some of the terms. Therefore, on the axis, these equations are replaced by specific boundary conditions, which are based on the symmetry properties of the fields (see \cite{LifschitzJCP2009} for details).

Compared to a 3D Cartesian calculation with $n_x\times n_y \times n_z$
grid cells, a quasi-cylindrical calculation with two modes ($l=0$ and $l=1$)
will require only $3 \,n_r \times n_z$ grid cells. Assuming $n_x=n_y=n_r=100$
as a typical transverse resolution, a quasi-cylindrical calculation is typically
over an order of magnitude less computationally demanding than its 3D Cartesian
equivalent.
