\input{../newcommands}

In general, the field is gathered from the mesh onto the macroparticles
using splines of the same order as for the current deposition $\mathbf{S}=\left(S_{x},S_{y},S_{z}\right)$.
Three variations are considered:
\begin{itemize}
\item ``momentum conserving'': fields are interpolated from the grid nodes
to the macroparticles using $\mathbf{S}=\left(S_{nx},S_{ny},S_{nz}\right)$
for all field components (if the fields are known at staggered positions,
they are first interpolated to the nodes on an auxiliary grid),
\item ``energy conserving (or Galerkin)'': fields are interpolated from
the staggered Yee grid to the macroparticles using $\left(S_{nx-1},S_{ny},S_{nz}\right)$
for $E_{x}$, $\left(S_{nx},S_{ny-1},S_{nz}\right)$ for $E_{y}$,
$\left(S_{nx},S_{ny},S_{nz-1}\right)$ for $E_{z}$, $\left(S_{nx},S_{ny-1},S_{nz-1}\right)$
for $B_{x}$, $\left(S_{nx-1},S_{ny},S_{nz-1}\right)$ for $B{}_{y}$
and$\left(S_{nx-1},S_{ny-1},S_{nz}\right)$ for $B_{z}$ (if the fields
are known at the nodes, they are first interpolated to the staggered
positions on an auxiliary grid),
\item ``uniform'': fields are interpolated directly form the Yee grid
to the macroparticles using $\mathbf{S}=\left(S_{nx},S_{ny},S_{nz}\right)$
for all field components (if the fields are known at the nodes, they
are first interpolated to the staggered positions on an auxiliary
grid).
\end{itemize}
As shown in \cite{BirdsallLangdon,HockneyEastwoodBook,LewisJCP1972},
the momentum and energy conserving schemes conserve momentum and energy
respectively at the limit of infinitesimal time steps and generally
offer better conservation of the respective quantities for a finite
time step. The uniform scheme does not conserve momentum nor energy
in the sense defined for the others but is given for completeness,
as it has been shown to offer some interesting properties in the modeling
of relativistically drifting plasmas \cite{GodfreyJCP2013}.
